\documentclass[12pt,a4paper]{article}
\usepackage{amsmath}
%\usepackage{amsthm}
\usepackage{acronym}
\usepackage[a4paper, hmargin={2.8cm, 2.8cm}, vmargin={2.5cm, 2.5cm}]{geometry}
\usepackage{eso-pic} % \AddToShipoutPicture
\usepackage{graphicx} % \includegraphics
\usepackage[utf8]{inputenc}
\usepackage{lmodern}
\usepackage{eurosym}
\usepackage{csquotes}
\usepackage{bookmark}
\usepackage{longtable}
\usepackage{array}
\usepackage{multirow}
\usepackage{times}
\usepackage{lingmacros}
\usepackage{color, colortbl}
\usepackage{tabularx}
\usepackage{pdfpages}
\usepackage{footnote}
\usepackage{microtype}
\usepackage{cleveref}
\usepackage{listings}
\usepackage{xcolor}
\usepackage{url}
\usepackage{tikz}
\definecolor{mygray}{rgb}{0.86,0.86,0.86}
\usepackage[defaultmono,scale=0.9]{droidmono}
\usepackage{bm}
\usepackage{algorithm}
\usepackage[noend]{algpseudocode}
\usepackage{amsfonts}
\usepackage[inline]{enumitem}
\usepackage{booktabs}
\usepackage{todonotes}
\usepackage{caption,subcaption}
\captionsetup{compatibility=false}
\usepackage{pifont}% http://ctan.org/pkg/pifont
\newcommand{\cmark}{\ding{51}}%
\newcommand{\xmark}{\ding{55}}%

\renewcommand{\vec}[1]{\bm{\mathrm{#1}}}
\usetikzlibrary{positioning}
\usetikzlibrary{calc}


\definecolor{cbred}{RGB}{215,25,28}
\definecolor{cborange}{RGB}{253,174,97}
\definecolor{cbyellow}{RGB}{255,255,191}
\definecolor{cbgreen}{RGB}{171,221,164}
\definecolor{cbblue}{RGB}{43,131,186}

% For highlighting
\makeatletter
\newenvironment{btHighlight}[1][]
{\begingroup\tikzset{bt@Highlight@par/.style={#1}}\begin{lrbox}{\@tempboxa}}
{\end{lrbox}\bt@HL@box[bt@Highlight@par]{\@tempboxa}\endgroup}

\newcommand\btHL[1][]{%
  \begin{btHighlight}[#1]\bgroup\aftergroup\bt@HL@endenv%
}
\def\bt@HL@endenv{%
  \end{btHighlight}%   
  \egroup
}
\newcommand{\bt@HL@box}[2][]{%
  \tikz[#1]{%
    \pgfpathrectangle{\pgfpoint{1pt}{0pt}}{\pgfpoint{\wd #2}{\ht #2}}%
    \pgfusepath{use as bounding box}%
    \node[anchor=base west, fill=cborange!30,outer sep=0pt,inner xsep=1pt, inner ysep=0pt, rounded corners=3pt, minimum height=\ht\strutbox+1pt,#1]{\raisebox{1pt}{\strut}\strut\usebox{#2}};
  }%
}
\makeatother

\lstset{language=Python,
  basicstyle=\small\ttfamily,
  keywordstyle=\color{cbblue},
  stringstyle=\color{cbred},
  morekeywords={with},
  moredelim=**[is][\btHL]{@}{@}
  }

\makesavenoteenv{tabular}
\makesavenoteenv{table}

\usepackage[backend=biber, style=authoryear, sorting=nty, maxcitenames=2]{biblatex}


\addbibresource{main.bib} %Imports bibliography file

\def \ColourPDF {cover/bilagscan-farve.pdf}

\title{
  \vspace{3.5cm}
  \LARGE{Experiences with using Probabilistic Programming for Voucher Feature
    Extraction at Skanned.com} \\
  \Large{Project Status Report}
}


 
\author{
  \Large{Ahmad Salim Al-Sibahi} \\
  \texttt{ahmad@bilagscan.dk}
}
\date{January 2019}


\begin{document} 
\pagenumbering{roman}
\AddToShipoutPicture*{\put(20,0){\includegraphics*{\ColourPDF}}}

\clearpage\maketitle
\thispagestyle{empty}

\newpage

\pagenumbering{arabic}

\section{Background}
Probabilistic Programming is an emergent field of machine learning, that
enriches general programming frameworks with probabilistic constructs from Bayesian reasoning.
The core idea is that one specifies probabilistic models that describe the
anticipated distribution of target data, and then these frameworks provide an
automated way to learn parameters of the model when new data is observed.

There are three key advantages to probabilistic programming over existing
machine learning technology: it is possible to directly incorporate \emph{domain
knowledge} using prior distributions, the constructed models allow for a systematic way to
\emph{quantify uncertainty}, and they are often directly
\emph{interpretable} by humans. All of these aspects are important for voucher
information extraction, which is the core business of Skanned.com.
Vouchers are almost always structured documents, where key information is
explicitly labelled and where there are many legal rules about how
information should be presented; the use of this domain knowledge is key to
achieving good results, which is why the existing system relies heavily on
hand-tuned heuristics. Quantifying uncertainty of a result is an important way
for the system to specify its trust in the results it provides and make sure
that customers only pay what is necessary: it is important for customers that
when the system states that the total amount is ``\$1000'', it is the right
amount and not ``\$100'' or ``\$10000''. Skanned.com provides a
human-based validation service, but such service is expensive and is a
bottleneck with regards to scalability; it is therefore important to only rely
on it is known to be necessary, which is not possible to do in a systematic way
with the current system.
Finally, if an error happens during information extraction, it is important that
the system is easy to debug and explain to customers, which is hard to do for
purely deep neural network-based architectures with millions of nuisance parameters.

The goal of the current Industrial PostDoc is to examine how to apply
probabilistic programming in practice for voucher scanning systems. This is
important both for the company, where getting a good solution can result in
significant savings and make the company a leading expert in machine learning,
and for science in general, since there are not many existing practical
applications of probabilistic programming and new experiences provide an
opportunity for improving the existing systems.

The PostDoc is mentored on the academic side by Dr. Thomas Hamelryck, who is an
expert in Bayesian data analysis and its applications in Bioinformatics, and
Dr. Fritz Henglein, who is an expert in programming languages and high-performance
compilation to GPUs. On the business side, the mentor is Dan Rose Johansen, who
is the Chief Operational Officer at BilagScan, and leads the daily operations.

\section{Achievements}
\subsection{Knowledge Building} During my PostDoc at Skanned.com I have worked
towards understanding how to apply probabilistic programming in practice, and
sharing such knowledge amongst colleagues and fellow academics.
Concretely:
\begin{itemize}
\item I have developed an understanding for Bayesian data analysis, including
  constructing probabilistic models, presents them and evaluating them.
\item I have gotten familiar with the wide range of inference techniques
  available for inference in probabilistic programs, gaining an understanding of
  their core theory and for the problems where they are applicable.
\item I have shared my knowledge about various probabilistic programming frameworks with
  my colleagues, and discussed how they can potentially be applied in practice.
\item I am a part of a weekly probabilistic programming discussion group at University of Copenhagen, where they are actively looking into
  how to use modern probabilistic programming frameworks for protein folding.
\end{itemize}

\subsection{Application} I have examined how we can
use probabilistic programming frameworks for voucher information extraction, and
developed various models aimed towards such goal.
\begin{itemize}
\item I have developed a model for grouping vouchers based on textual and visual
  features\footnote{In collaboration with my mentors Fritz and Thomas},
  that relies on latent Dirichlet allocation (LDA), which is a popular
  probabilistic programming technique. The grouping is useful for developing
  specialized algorithms for similar sets of vouchers and thus increase
  precision in the information extraction. This model was found useful by
  Skanned.com and is therefore planned to soon come in production.
\item I have developed a series of probabilistic models from scratch for
  keyword-based feature localization. The problem is challenging to encode
  because of the varying number of keywords and features between each document,
  but initial results were promising in showing its potential use in practice:
  80\% of the time the target feature was the expected one, and 99\% of the time
  it was within the 95\% confidence interval.
\item I am currently working with a more ambitious model that tries to assigns
  keywords to features explicitly, to make inference more precise. The model can
  also be extended to allow locating more features in the future, as well as possibly identify potential
  keywords.
\item I have developed experience with various probabilistic programming
  frameworks like PyMC3, Pyro and Infer.Net, and their underlying technologies
  like Theano and PyTorch. This included getting experiences with neural network
  architectures that can potentially be used in the models or for inference.
  Furthermore, I have actively submitted bug reports, bug fixes and features to
  these languages.
\end{itemize}

\subsection{Dissemination}
I have tried to share my experiences using Probabilistic Programming as part of
my Industrial PostDoc at Skanned.com, in order to generate excitement for the
area. This is also a good branding opportunity for Skanned.com to position
themselves as a cutting edge startup in artificial intelligence (AI).
\begin{itemize}
\item I have presented our work at the first international conference on
  probabilistic programming (PROBPROG 2018), which was held at Massachusetts Institute of
  Technology (MIT). This provided opportunity to learn about the latest
  technology within probabilistic programming by world's top universities and
  major IT companies (Microsoft/Facebook/Google/Uber/Amazon/Oracle etc.)
\item I have presented our plans for probabilistic programming at a public event
  organized by Skanned.com on Machine Learning and Artificial Intelligence
  (called Masters of AI/ML 2018). I was rated as one of the top speakers at this
  event by the audience, which was around 200 people.
\item I have started a meet-up on probabilistic programming, to provide more
  focused technical discussions on the area. The meet-up is a collaboration
  between Skanned.com, University of Copenhagen and Hypefactors, and has already
  after one session over 70 members, which is great for such new techology. We
  already have scheduled multiple sessions and in talk to have exciting speakers from
  academia and industry in the future.
\end{itemize}
In general, it should be mentioned that probabilistic programming as a
technology is getting a lot of excitement. This is also felt by my mentors
Thomas and Fritz, when they present and discuss this work with other academics
and people from industry.

\section{Challenges}

\section{Prospectives}

\medskip

\printbibliography[
heading=bibintoc,
title={Bibliography}
] 

\clearpage

\end{document}

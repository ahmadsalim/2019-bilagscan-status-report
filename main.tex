\documentclass[12pt,a4paper]{article}
\usepackage{amsmath}
%\usepackage{amsthm}
\usepackage{acronym}
\usepackage[a4paper, hmargin={2.8cm, 2.8cm}, vmargin={2.5cm, 2.5cm}]{geometry}
\usepackage{eso-pic} % \AddToShipoutPicture
\usepackage{graphicx} % \includegraphics
\usepackage[utf8]{inputenc}
\usepackage{lmodern}
\usepackage{eurosym}
\usepackage{csquotes}
\usepackage{bookmark}
\usepackage{longtable}
\usepackage{array}
\usepackage{multirow}
\usepackage{times}
\usepackage{lingmacros}
\usepackage{color, colortbl}
\usepackage{tabularx}
\usepackage{pdfpages}
\usepackage{footnote}
\usepackage{microtype}
\usepackage{cleveref}
\usepackage{listings}
\usepackage{xcolor}
\usepackage{url}
\usepackage{tikz}
\definecolor{mygray}{rgb}{0.86,0.86,0.86}
\usepackage[defaultmono,scale=0.9]{droidmono}
\usepackage{bm}
\usepackage{algorithm}
\usepackage[noend]{algpseudocode}
\usepackage{amsfonts}
\usepackage[inline]{enumitem}
\usepackage{booktabs}
\usepackage{todonotes}
\usepackage{caption,subcaption}
\captionsetup{compatibility=false}
\usepackage{pifont}% http://ctan.org/pkg/pifont
\newcommand{\cmark}{\ding{51}}%
\newcommand{\xmark}{\ding{55}}%

\renewcommand{\vec}[1]{\bm{\mathrm{#1}}}
\usetikzlibrary{positioning}
\usetikzlibrary{calc}


\definecolor{cbred}{RGB}{215,25,28}
\definecolor{cborange}{RGB}{253,174,97}
\definecolor{cbyellow}{RGB}{255,255,191}
\definecolor{cbgreen}{RGB}{171,221,164}
\definecolor{cbblue}{RGB}{43,131,186}

% For highlighting
\makeatletter
\newenvironment{btHighlight}[1][]
{\begingroup\tikzset{bt@Highlight@par/.style={#1}}\begin{lrbox}{\@tempboxa}}
{\end{lrbox}\bt@HL@box[bt@Highlight@par]{\@tempboxa}\endgroup}

\newcommand\btHL[1][]{%
  \begin{btHighlight}[#1]\bgroup\aftergroup\bt@HL@endenv%
}
\def\bt@HL@endenv{%
  \end{btHighlight}%   
  \egroup
}
\newcommand{\bt@HL@box}[2][]{%
  \tikz[#1]{%
    \pgfpathrectangle{\pgfpoint{1pt}{0pt}}{\pgfpoint{\wd #2}{\ht #2}}%
    \pgfusepath{use as bounding box}%
    \node[anchor=base west, fill=cborange!30,outer sep=0pt,inner xsep=1pt, inner ysep=0pt, rounded corners=3pt, minimum height=\ht\strutbox+1pt,#1]{\raisebox{1pt}{\strut}\strut\usebox{#2}};
  }%
}
\makeatother

\lstset{language=Python,
  basicstyle=\small\ttfamily,
  keywordstyle=\color{cbblue},
  stringstyle=\color{cbred},
  morekeywords={with},
  moredelim=**[is][\btHL]{@}{@}
  }

\makesavenoteenv{tabular}
\makesavenoteenv{table}

\usepackage[backend=biber, style=authoryear, sorting=nty, maxcitenames=2]{biblatex}


\addbibresource{main.bib} %Imports bibliography file

\def \ColourPDF {cover/bilagscan-farve.pdf}

\title{
  \vspace{3.5cm}
  \LARGE{Experiences with using Probabilistic Programming for Voucher Feature
    Extraction at Skanned.com} \\
  \Large{Project Status Report}
}


 
\author{
  \Large{Ahmad Salim Al-Sibahi} \\
  \texttt{ahmad@bilagscan.dk}
}
\date{January 2019}


\begin{document} 
\pagenumbering{roman}
\AddToShipoutPicture*{\put(20,0){\includegraphics*{\ColourPDF}}}

\clearpage\maketitle
\thispagestyle{empty}

\newpage

\pagenumbering{arabic}

\section{Introduction}

\subsection{Background}
Probabilistic Programming is an emergent field of machine learning, that
enriches general programming frameworks with probabilistic constructs from Bayesian reasoning.
The core idea is that one specifies probabilistic models that describe the
anticipated distribution of target data, and then these frameworks provide an
automated way to learn parameters of the model when new data is observed.

There are three key advantages to probabilistic programming over existing
machine learning technology: it is possible to directly incorporate \emph{domain
knowledge} using prior distributions, the constructed models allow for a systematic way to
\emph{quantify uncertainty}, and they are often directly
\emph{interpretable} by humans. All of these aspects are important for voucher
information extraction, which is the core business of Skanned.com.
Vouchers are almost always structured documents, where key information is
explicitly labelled and where there are many legal rules about how
information should be presented; the use of this domain knowledge is key to
achieving good results, which is why the existing system relies heavily on
hand-tuned heuristics. Quantifying uncertainty of a result is an important way
for the system to specify its trust in the results it provides and make sure
that customers only pay what is necessary: it is important for customers that
when the system states that the total amount is ``\$1000'', it is the right
amount and not ``\$100'' or ``\$10000''. Skanned.com provides a
human-based validation service, but such service is expensive and is a
bottleneck with regards to scalability; it is therefore important to only rely
on it is known to be necessary, which is not possible to do in a systematic way
with the current system.
Finally, if an error happens during information extraction, it is important that
the system is easy to debug and explain to customers, which is hard to do for
purely deep neural network-based architectures with millions of nuisance parameters.

The goal of the current Industrial PostDoc is to examine how to apply
probabilistic programming in practice for voucher scanning systems. This is
important both for the company, where getting a good solution can result in
significant savings and make the company a leading expert in machine learning,
and for science in general, since there are not many existing practical
applications of probabilistic programming and new experiences provide an
opportunity for improving the existing systems.

\subsection{Achievements}

\subsection{Challenges}

\subsection{Plans}


\section{Probabilistic Programming}

\section{Models for Voucher Processing}

\section{PP in Broader Context}

\section{Perspectives}

\medskip

\printbibliography[
heading=bibintoc,
title={Bibliography}
] 

\clearpage

\end{document}

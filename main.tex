\documentclass[12pt,a4paper]{article}
\usepackage[hyphens]{url}
\usepackage{amsmath}
%\usepackage{amsthm}
\usepackage{acronym}
\usepackage[a4paper, hmargin={2.8cm, 2.8cm}, vmargin={2.5cm, 2.5cm}]{geometry}
\usepackage{eso-pic} % \AddToShipoutPicture
\usepackage{graphicx} % \includegraphics
\usepackage[utf8]{inputenc}
\usepackage{lmodern}
\usepackage{eurosym}
\usepackage{csquotes}
\usepackage{bookmark}
\usepackage{longtable}
\usepackage{array}
\usepackage{multirow}
\usepackage{times}
\usepackage{lingmacros}
\usepackage{color, colortbl}
\usepackage{tabularx}
\usepackage{pdfpages}
\usepackage{footnote}
\usepackage{microtype}
\usepackage{cleveref}
\usepackage{listings}
\usepackage{xcolor}
\usepackage{tikz}
\definecolor{mygray}{rgb}{0.86,0.86,0.86}
\usepackage[defaultmono,scale=0.9]{droidmono}
\usepackage{bm}
\usepackage{algorithm}
\usepackage[noend]{algpseudocode}
\usepackage{amsfonts}
\usepackage[inline]{enumitem}
\usepackage{booktabs}
\usepackage{todonotes}
\usepackage{caption,subcaption}
\captionsetup{compatibility=false}
\usepackage{pifont}% http://ctan.org/pkg/pifont
\newcommand{\cmark}{\ding{51}}%
\newcommand{\xmark}{\ding{55}}%

\renewcommand{\vec}[1]{\bm{\mathrm{#1}}}
\usetikzlibrary{positioning}
\usetikzlibrary{calc}


\definecolor{cbred}{RGB}{215,25,28}
\definecolor{cborange}{RGB}{253,174,97}
\definecolor{cbyellow}{RGB}{255,255,191}
\definecolor{cbgreen}{RGB}{171,221,164}
\definecolor{cbblue}{RGB}{43,131,186}

% For highlighting
\makeatletter
\newenvironment{btHighlight}[1][]
{\begingroup\tikzset{bt@Highlight@par/.style={#1}}\begin{lrbox}{\@tempboxa}}
{\end{lrbox}\bt@HL@box[bt@Highlight@par]{\@tempboxa}\endgroup}

\newcommand\btHL[1][]{%
  \begin{btHighlight}[#1]\bgroup\aftergroup\bt@HL@endenv%
}
\def\bt@HL@endenv{%
  \end{btHighlight}%   
  \egroup
}
\newcommand{\bt@HL@box}[2][]{%
  \tikz[#1]{%
    \pgfpathrectangle{\pgfpoint{1pt}{0pt}}{\pgfpoint{\wd #2}{\ht #2}}%
    \pgfusepath{use as bounding box}%
    \node[anchor=base west, fill=cborange!30,outer sep=0pt,inner xsep=1pt, inner ysep=0pt, rounded corners=3pt, minimum height=\ht\strutbox+1pt,#1]{\raisebox{1pt}{\strut}\strut\usebox{#2}};
  }%
}
\makeatother

\lstset{language=Python,
  basicstyle=\small\ttfamily,
  keywordstyle=\color{cbblue},
  stringstyle=\color{cbred},
  morekeywords={with},
  moredelim=**[is][\btHL]{@}{@}
  }

\makesavenoteenv{tabular}
\makesavenoteenv{table}

\usepackage[backend=biber, style=authoryear, sorting=nty, maxcitenames=2]{biblatex}


\addbibresource{main.bib} %Imports bibliography file

\def \ColourPDF {cover/bilagscan-farve.pdf}

\title{
  \vspace{3.5cm}
  \LARGE{Experiences with using Probabilistic Programming for Voucher Feature
    Extraction at Skanned.com} \\
  \Large{Project Status Report}
}


 
\author{
  \Large{Ahmad Salim Al-Sibahi} \\
  \texttt{ahmad@bilagscan.dk}
}
\date{January 2019}


\begin{document} 
\pagenumbering{roman}
\AddToShipoutPicture*{\put(20,0){\includegraphics*{\ColourPDF}}}

\clearpage\maketitle
\thispagestyle{empty}

\newpage

\pagenumbering{arabic}

\section{Background}
Probabilistic Programming is an emergent field of machine learning that
enriches general programming frameworks with probabilistic constructs from Bayesian reasoning.
The core idea is that one specifies probabilistic models that describe the
anticipated distribution of target data, and then these frameworks provide an
automated way to learn parameters of the model when new data is observed.

There are three key advantages to probabilistic programming over existing
machine learning technology: it is possible to directly incorporate \emph{domain
knowledge} using prior distributions, the constructed models allow for a systematic way to
\emph{quantify uncertainty}, and they are often directly
\emph{interpretable} by humans. All of these aspects are important for voucher
information extraction, which is the core business of Skanned.com.
Vouchers are almost always structured documents, where key information is
explicitly labelled and where there are many legal rules about how
information should be presented; the use of this domain knowledge is key to
achieving good results, which is why the existing system relies heavily on
hand-tuned heuristics. Quantifying uncertainty of a result is an important way
for the system to specify its trust in the results it provides and make sure
that customers only pay what is necessary: it is important for customers that
when the system states that the total amount is ``\$1000'' then it is correct,
and not ``\$100'' or ``\$10000'' instead. Skanned.com provides a
human-based validation service, but such service is expensive and is a
bottleneck with regards to scalability: it is therefore important to only use
it when necessary, which is hard to determine with the current system.
Finally, when an error happens during information extraction, it is important that
the system is easy to debug and explain to customers, which is hard to do for
deep neural network-based architectures with millions of nuisance parameters.

The goal of the current Industrial PostDoc is to examine how to apply
probabilistic programming in practice for voucher scanning systems. This is
important both for the company, where getting a good solution can result in
significant savings and make the company a leading expert in machine learning,
and for science in general, since there are not many existing practical
applications of probabilistic programming and new experiences provide an
opportunity for improving the existing systems.

The PostDoc is mentored on the academic side by Dr. Thomas Hamelryck, who is an
expert in Bayesian data analysis and its applications in Bioinformatics, and
Dr. Fritz Henglein, who is an expert in programming languages and high-performance
compilation to GPUs. On the business side, the mentor is Dan Rose Johansen, who
is the Chief Operational Officer at BilagScan, and leads the daily operations.

\section{Achievements}
\subsection{Knowledge Building} During my PostDoc at Skanned.com I have worked
towards understanding how to apply probabilistic programming in practice, and
sharing such knowledge amongst colleagues and fellow academics.
Concretely:
\begin{itemize}
\item I developed an understanding of Bayesian data analysis, including
  constructing probabilistic models, presenting them and evaluating them.
\item I got familiar with the wide range of inference techniques
  available for probabilistic programs, understood
  their core theory and the problems for which they are applicable.
\item I shared my knowledge about various probabilistic programming frameworks with
  my colleagues, and discussed how to apply them in practice.
\item I am a part of a weekly probabilistic programming discussion group at University of Copenhagen, where they are actively looking into
  how to use modern probabilistic programming frameworks for solving problems in academia in general.
\end{itemize}

\subsection{Current Results and Performance}
I examined how we can
use probabilistic programming frameworks for voucher information extraction, and
developed various models aimed towards such goal.
\begin{itemize}
\item I developed a model for grouping vouchers based on textual and visual
  features\footnote{In collaboration with my mentors Fritz and Thomas},
  that relies on latent Dirichlet allocation (LDA), which is a Bayesian model
  that identifies a common set of ``topics'' that summarize a set of documents.
  The grouping is useful for developing
  specialized algorithms for similar sets of vouchers and thus increase
  precision in the information extraction.
\item I developed a series of probabilistic models from scratch for
  keyword-based feature localization. The problem is challenging to encode
  because of the varying number of keywords and features between each document,
  but initial results were promising in showing its potential use in practice:
  \begin{itemize}
  \item 80\% of the time the target feature was the expected one, and 99\% of the time
    it was within the 95\% confidence interval.
  \item Due to the quantification of uncertainty, it can exclude irrelevant
    features systematically without losing precision. The ad-hoc methods
    sometimes excluded correct features arbitrarily, even though our technique
    was certain that it was the only possible correct choice.
  \item I have optimized the code so that it can take around 1500 high-quality Hamiltonian Monte Carlo
    (HMC)\footnote{A sampling technique that relies on gradient information to
      provide quicker convergence towards the true posterior distribution.}
    samples in 2-3 seconds on average, which is excellent for such an
    expressive inference technique.
  \end{itemize}
\item To make inference more precise, I am currently working with a more ambitious model that tries to assigns
  keywords to features explicitly. The model can
  also be extended to allow locating more features in the future, as well as possibly identify potential
  keywords.
\item I developed experience with various probabilistic programming
  frameworks like PyMC3, Pyro and Infer.Net, and their underlying technologies
  like Theano and PyTorch. This included getting experiences with neural network
  architectures that can potentially be used in the models or for inference.
  Furthermore, I have actively submitted bug reports, bug fixes and features to
  these languages.
\end{itemize}

\subsection{Production}
The grouping model that we have developed during this project has already been
rewritten by the developers at Skanned.com to be integrated in production, and
is expected to come in production soon.
The model has been found very valuable by the company since it allows improving their
current system in multiple ways:
\begin{itemize}
\item It can be used as input to the template-based voucher information
  extraction system, which relies on the existing sets of processed vouchers to extract
  information from new similar vouchers. The current way that processed vouchers
  are selected now are using the VAT registration number, but that is limited
  compared to similarity scores given by the grouping model: the
  VAT registration number is specifically tied to a customer, so it will
  miss new customers that have similar vouchers (e.g. printed using the same
  system), and it is less robust when the customer changes the layout of their vouchers.
\item It can be used as an input to future machine learning techniques developed
  at Skanned.com (including probabilistic programming), in order to improve
  precision of information extraction by allowing learning of parameters that
  are specific to groups of similar vouchers.
\end{itemize}

\subsection{Dissemination}
I have tried to share my experiences using Probabilistic Programming as part of
my Industrial PostDoc at Skanned.com, in order to generate excitement for the
area. This is also a good branding opportunity for Skanned.com to position
themselves as a cutting edge startup in artificial intelligence (AI).
\begin{itemize}
\item I have presented our work at the first international conference on
  probabilistic programming (PROBPROG 2018), which was held at Massachusetts Institute of
  Technology (MIT). This provided opportunity to learn about the latest
  technology within probabilistic programming by world's top universities and
  major IT companies (Microsoft/Facebook/Google/Uber/Amazon/Oracle etc.)
\item I have presented our plans for probabilistic programming at a public event
  organized by Skanned.com on Machine Learning and Artificial Intelligence
  (called Masters of AI/ML 2018). I was rated as one of the top speakers at this
  event by the audience, which was around 200 people.
\item I have started a meet-up on probabilistic programming, to provide more
  focused technical discussions on the area. The meet-up is a collaboration
  between Skanned.com, University of Copenhagen and Hypefactors, and has already
  after one session over 75 members, which is great for such new techology. We
  already have scheduled multiple sessions and in talk to have exciting speakers from
  academia and industry (e.g., Facebook, Nordea) in the future.
\end{itemize}
In general, it should be mentioned that probabilistic programming as a
technology is getting a lot of excitement. This is also felt by my mentors
Thomas and Fritz, when they present and discuss this work with other academics
and people from industry.

\section{Challenges}
Being a first mover in a new field of technology like probabilistic programming
has its clear rewards, but also comes with its own set of challenges, both
anticipated and unexpected. I will here describe some of the challenges
encountered during the project and how I strived to solve them.

\subsection{Theoretical Challenges}
Today, there are few applications of general probabilistic programming in
industry, and none we know of within the area of voucher processing. While the
development of probabilistic programming framework is also done by research arms
of industrial companies, most papers are written towards academics using
academic language. There is still a reasonable gap between theory and practice,
which we as part of this project hope to fill.
\begin{itemize}
\item Probabilistic Programming intersects many areas of theoretical and applied
  mathematics. Knowledge is required at least in the fields of probability
  theory, advanced calculus and programming language semantics (including
  category theory), and sometimes even areas like statistics, information
  theory, measure theory and real analysis.
  Some of these areas I had to refresh my knowledge in---I had
  not made heavy use of calculus or probability theory since high-school---and
  some I had to learn from scratch, like basic measure theory.
\item The promise of probabilistic programming, where the user provides a model
  and where inference is fully automatic is perhaps not quite achieved. It seems
  certainly easier to do Bayesian modeling now than few years back, but
  knowledge of the particular inference methods is still necessary to construct
  models that work well. For example, sampling methods require good geometric
  ergodicity\footnote{Distributions should not have heavy tails or complex
modes.}, variational methods require estimators with low variance, and
  expectation propagation and similar algorithms require priors to be conjugate.
\item Working with mixed discrete-continuous models is still a challenge from the
  side of inference. Discrete models make differentiation-based methods harder
  to use; those algorithms that rely on both differentiation and discrete
  methods work only well for small number of discrete choices, and those that do
  not rely on differentiation only work with fixed sets of continuous distributions.
\item The maturity of existing probabilistic programming framework is still low.
  Frameworks like Anglican or Venture still seem to be at the academic
  stage, while more industry targeted frameworks like PyMC3, Pyro and Infer.Net
  still have limitations for more complex models. During the project, I had to
  debug\,\footnote{See ``Getting auto-enumeration working with transformed
    distributions'' (\url{https://forum.pyro.ai/t/getting-auto-enumeration-working-with-transformed-distributions/440a}),
``NaN occured in optimization'' (\url{https://discourse.pymc.io/t/nan-occured-in-optimization-in-a-vonmises-mixture-model/1296}),
``[bug] --jit option fails for HMM example''
(\url{https://github.com/uber/pyro/issues/1435}),
``pm.sample\_prior\_predictive
fails when model contains a mixture-based distribution''
(\url{https://github.com/pymc-devs/pymc3/issues/3101}), and ``Updated mixture model breaks with multi-dimension mixtures''
(\url{https://github.com/pymc-devs/pymc3/issues/3044})
}
and profile\,\footnote{See ``Profiling TraceEnumELBO''
  (\url{https://forum.pyro.ai/t/invalid-index-in-gather-for-model-relying-on-auto-enumeration/479/5})} the frameworks themselves to identify bugs and limitations,
  discuss the issues with the framework developers and contribute
  code\,\footnote{See ``Working on component-wise transformations that mimic
    torch.cat and torch.stack''
    (\url{https://github.com/pytorch/pytorch/pull/11868}), ``Add documentation
    for transform functions''
    (\url{https://github.com/pymc-devs/pymc3/pull/3192}), ``Add site information
    for possible exception thrown when calculating logp''
    (\url{https://github.com/uber/pyro/pull/1509}) and ``Clarify phrasing of
    installation instructions with regards to running examples'' (\url{https://github.com/uber/pyro/pull/1533})} to fix
  some of these issues.
\item The performance and scalability of current probabilistic frameworks still
  needs improvement. We have so far not been able to train with more than around 1000
  vouchers in reasonable time for more complex models, even though the available
  data set is orders of magnitude larger. Even though most framework support
  running models on the GPU, it is hard to achieve good performance improvements
  and sometimes the overhead is larger than running on CPUs directly.
\end{itemize}

\subsection{Practical Issues}
In the beginning of the project there were some practical issues at the
University about the time of getting the budgeted research machine, which have now been solved. However, these issues made it a
non-negligible impact on the project and caused some delays.

\section{Future Directions}
During this project there have been many challenges that have been overcome and
many achievements that have been made.
For the future, there are a few things which we want to work on:
\begin{itemize}
\item Finish the keyword assignment-based algorithm and publish the current
  results (including grouping) in a suitable conference (e.g. ECML PKDD). This
  might require some time to get working optimally with inference, since the
  model is complex and has many local optima, but can possibly be extended with
  more directions in the future.
  \begin{itemize}
  \item Alternatively, work on extending the
  keyword-based localization that provided initially good results with more
  features to increase accuracy; this can provide a faster road to currently useful
  results, but is less extensible in the future.
  \end{itemize}
\item Tackle new problem in voucher processing with probabilistic programming. A
  potential problem is extraction of product lines, where the current algorithm of
  Skanned.com lacks behind. Product lines describe information about each individual item
  that is purchased on a voucher, and they are complex to model because of the
  variation of headings across different voucher types and variation of the number of
  lines even between vouchers of the same type.
\item Use practical experience gathered during the project so far
  to collaborate with the Futhark group at University of Copenhagen to improve
  see how to scale existing probabilistic programming techniques to run much
  more efficiently on the GPU and allow for bigger data-sets to be handled.
\end{itemize}

\medskip

\printbibliography[
heading=bibintoc,
title={Bibliography}
] 

\clearpage

\end{document}
